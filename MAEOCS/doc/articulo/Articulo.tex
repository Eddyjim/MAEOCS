\documentclass[journal]{IEEEtran}

\usepackage[utf8]{inputenc}
%\usepackage{program}
\usepackage[spanish,activeacute]{babel}
\usepackage[pdftex]{graphics}   % Para incluir EPS.
% %\usepackage{rotating}   % Para rotar tablas.
\usepackage{array}      % Para hacer tablas raras.
\usepackage{epsfig}
%\usepackage{amstex}      % Debe estar despues de babel !!!
\usepackage{amsmath}      % Debe estar despues de babel !!!
\usepackage{cite}
\usepackage{listings}

\newcommand{\eci}{\textsf{Escuela Colombiana de Ingenier�a Julio Garavito~}}
\newcommand{\eciu}{\textsf{Escuela Colombiana de Ingenier�a}}
\newcommand{\Rlb}{\textsf{Rodrigo L�pez B.}}
\newcommand{\andr}{\texttt{Android}}
\newcommand{\andrs}{\texttt{Android~}}
\newcommand{\asdk}{\texttt{Android SDK~}}
\newcommand{\mae}{\texttt{MAEOCS}}
\newcommand{\amnf}{\texttt{AndroidManifest.xml}}
\newcommand{\amnfs}{\texttt{AndroidManifest.xml~}}
\newcommand{\dalvm}{\texttt{Dalvik virtual machine}}
\newcommand{\cp}{\texttt{Content Providers}}
\newcommand{\rem}{\texttt{Resource Manager}}
\newcommand{\rems}{\texttt{Resource Manager}}
\newcommand{\am}{\texttt{Activity Manager}}
\newcommand{\funs}{\texttt{Fun Menu~}}
\newcommand{\fun}{\texttt{Fun Menu}}
\newcommand{\aosp}{\texttt{The Android Open Source Project}}
\newcommand{\aosps}{\texttt{The Android Open Source Project~}}
\newcommand{\apf}{\texttt{The Android Application Framework~}}


\begin{document}
%
% paper title
% can use linebreaks \\ within to get better formatting as desired
\title{MAEOCS: Mobile Application for Easy Orientation in Confined Spaces}
\date{Octubre 2012}
\author{Carlos Gait�n Mora,~\IEEEmembership{Estudiante,~\eci}
  \and Edward Jim�nez Mart�nez,~\IEEEmembership{Estudiante,~\eci}
}

% The paper headers
\markboth{MAEOCS,~Art�culo de Proyecto de Grado, Diciembre~2012}%
{Shell \MakeLowercase{\textit{et al.}}: Bare Demo of IEEEtran.cls for Journals}

% make the title area
\maketitle


\begin{abstract}
%\boldmath
Aaasw
\end{abstract}

\begin{IEEEkeywords}
Android, Java, GPS.
\end{IEEEkeywords}

\IEEEpeerreviewmaketitle

\section{Introducci�n}

\IEEEPARstart{H}{oy} en d�a es f�cil evidenciar c�mo los planteles p�blicos y privados van en crecimiento, ampliando sus zonas en incluso dispersadoras en diferentes puntos de un mismo sector, por ello las personas que son nuevas en estos ambientes, o que poseen mala memoria,  alguna discapacidad cognitiva, o tenga problemas para comunicarse a otros, lo cual les impide ubicarse de una forma f�cil y se ven afectadas cuando pierden el tiempo mientras miran mapas que son poco claros. 

Teniendo en cuenta que en esta era tecnol�gica, los dispositivos m�viles cuentan con una capacidad de procesamiento que d�a a d�a aumenta. Las limitaciones que se ten�a para resolver problemas de la vida cotidiana a trav�s de un computador se hacen cada vez mas peque�as. Esta tecnolog�a a impulsado un nuevo modelo de desarrollo de aplicaciones para dispositivos m�viles. Se han desarrollado frameworks y arquitecturas que facilitan el desarrollo e interacci�n con distintas tecnolog�as. 

En este documento se describe el desarrollo de una aplicaci�n que permite orientar a una persona a su destino dentro de un sitio cerrado donde un GPS convencional no pueda hacerlo, a trav�s de su celular. Esta aplicaci�n requiere el desarrollo de una aplicaci�n que permita la creaci�n de un mapa que pueda usarse en la aplicaci�n m�vil. Se describe la metodolog�a que se utiliza para encontrar los caminos que muestra la aplicaci�n. Y se mostrar�n los resultados obtenidos tanto a nivel t�cnico de las aplicaciones, como de los resultados funcionales de la aplicaci�n. 

\section{Conclusiones}
Ya que \andr es actualmente el sistema operativo para dispositivos moviles mas usado en el mundo, ofrece una ventaja hablando del numero de usuarios potenciales para una aplicaci�n. Ya que el objetivo de la aplicaci�n es ayudar a ubicarse en lugares de dificil orientaci�n, hace que el proyecto tenga su foco en la parte social, y para producir un mayor impacto se debe alcanzar la mayor cantidad de personas. Es por esto que \andr es la plataforma adecuada para hacerlo.
El desarrollo de la aplicaci�n permit�o la creaci�n de guias para aprendizaje y desarrollo de aplicaciones en el mismo sistema, beneficiando no solo a los usuarios de la aplicaci�n sino a su vez a futuros desarrolladores. Durante la investigaci�n para dar soluci�n del problema de los caminos cortos, se opt� por usar un algoritmo de busqueda conocido por su eficiencia, esto permite que la aplicaci�n se pueda usar en la mayor�a de dispositivos, sin necesidad de tener que usar dispositivos de �ltima tecnologia, y as� poder alcanzar mayor cantidad de usuarios.




\renewcommand{\refname}{References}
\begin{thebibliography}{50}
\bibitem{And} Android Developers Official Website.\\
\begin{scriptsize}  \verb|http://developer.android.com/guide/basics/what-is-android.html|\end{scriptsize}
\bibitem{WikiAnd}
\begin{scriptsize}  \verb|http://en.wikipedia.org/wiki/Android_(operating_system)|\end{scriptsize}
\bibitem{WikiOp}
\begin{scriptsize}  \verb|http://en.wikipedia.org/wiki/Open_Handset_Alliance|\end{scriptsize}
\bibitem{dalvik}
\begin{scriptsize}  \verb|http://en.wikipedia.org/wiki/Dalvik_(software)|\end{scriptsize}
\bibitem{deve}
\begin{scriptsize}  \verb|http://thedevelopersinfo.com/2009/11/17/using-assets-in-android/|\end{scriptsize}
\bibitem{dalvikvm}
\begin{scriptsize}  \verb|http://www.dalvikvm.com/|\end{scriptsize}
\bibitem{soa}
\begin{scriptsize}  \verb|http://developer.android.com/reference/android/util/Log.html| \end{scriptsize}`
\begin{scriptsize}Android in Action, Second Edition, W. Frank Ableson, Robi Sen, Chris King, Manning \end{scriptsize}
\bibitem{beginAnd}
\begin{scriptsize}Beginning Android 2, Mark L. Murphy, Apress \end{scriptsize}
\bibitem{andCook}
\begin{scriptsize}The Android Developer's Cookbook Building Applications with the Android SDK, James Steele, Nelson To, 
Addison-Wesley \end{scriptsize}
\bibitem{rest}

\end{thebibliography}

\end{document}
