\documentclass[journal]{IEEEtran}

\usepackage[T1]{fontenc}
\usepackage[utf8]{inputenc}
\usepackage{lmodern}
%\usepackage[spanish,activeacute]{babel}
\usepackage[pdftex]{graphics}   % Para incluir EPS.
%\usepackage{rotating}   % Para rotar tablas.
\usepackage{array}        % Para hacer tablas.
\usepackage{epsfig}
\usepackage{amsmath}      % Debe estar despues de babel !!!
\usepackage{cite}
\usepackage{listings}

\newcommand{\eci}{\textsf{Escuela Colombiana de Ingeniería Julio Garavito}}
\newcommand{\eciu}{\textsf{Escuela Colombiana de Ingeniería}}
\newcommand{\Rlb}{\textsf{Rodrigo López B.}}
\newcommand{\andr}{\texttt{Android~}}

\begin{document}

\title{MAEOCS: Mobile Application for Easy Orientation in Confined Spaces}
\date{Octubre 2012}
\author{Carlos~I.~Gaitán,~\IEEEmembership{Estudiante,~\eci,} y Edward~H.~Jiménez,~\IEEEmembership{Estudiante,~\eci.}}

\markboth{MAEOCS, Artículo de Proyecto de Grado, Diciembre 2012}
{Shell \MakeLowercase{\textit{et al.}}: Bare Demo of IEEEtran.cls for Journals}


\maketitle

\renewcommand*\contentsname{Contenido}
\tableofcontents

\begin{abstract}
\end{abstract}

\begin{IEEEkeywords}
Android, Java, GPS.
\end{IEEEkeywords}

\IEEEpeerreviewmaketitle

\section{Introducción}

\IEEEPARstart{H}{}oy en día es fácil evidenciar cómo los planteles públicos y privados van en crecimiento, ampliando sus zonas en incluso dispersadoras en diferentes puntos de un mismo sector, por ello las personas que son nuevas en estos ambientes, o que poseen mala memoria,  alguna discapacidad cognitiva, o tenga problemas para comunicarse a otros, lo cual les impide ubicarse de una forma fácil y se ven afectadas cuando pierden el tiempo mientras miran mapas que son poco claros. 

Teniendo en cuenta que en esta era tecnológica, los dispositivos móviles cuentan con una capacidad de procesamiento que día a día aumenta. Las limitaciones que se tenía para resolver problemas de la vida cotidiana a través de un computador se hacen cada vez mas pequeñas. Esta tecnología a impulsado un nuevo modelo de desarrollo de aplicaciones para dispositivos móviles. Se han desarrollado frameworks y arquitecturas que facilitan el desarrollo e interacción con distintas tecnologías. 

En este documento se describe el desarrollo de una aplicación que permite orientar a una persona a su destino dentro de un sitio cerrado donde un GPS convencional no pueda hacerlo, a través de su celular. Esta aplicación requiere el desarrollo de una aplicación que permita la creación de un mapa que pueda usarse en la aplicación móvil. Se describe la metodología que se utiliza para encontrar los caminos que muestra la aplicación. Y se mostrarán los resultados obtenidos tanto a nivel técnico de las aplicaciones, como de los resultados funcionales de la aplicación. 

\section{Conclusiones}
Ya que \andr es actualmente el sistema operativo para dispositivos moviles mas usado en el mundo, ofrece una ventaja hablando del numero de usuarios potenciales para una aplicación. Ya que el objetivo de la aplicación es ayudar a ubicarse en lugares de dificil orientación, hace que el proyecto tenga su foco en la parte social, y para producir un mayor impacto se debe alcanzar la mayor cantidad de personas. Es por esto que \andr es la plataforma adecuada para hacerlo.
El desarrollo de la aplicación permitío la creación de guias para aprendizaje y desarrollo de aplicaciones en el mismo sistema, beneficiando no solo a los usuarios de la aplicación sino a su vez a futuros desarrolladores. Durante la investigación para dar solución del problema de los caminos cortos, se optó por usar un algoritmo de busqueda conocido por su eficiencia, esto permite que la aplicación se pueda usar en la mayoría de dispositivos, sin necesidad de tener que usar dispositivos de última tecnologia, y así poder alcanzar mayor cantidad de usuarios.




\renewcommand{\refname}{Referencias}
\begin{thebibliography}{50}
\bibitem{And} Android Developers Official Website.\\
\begin{scriptsize}  \verb|http://developer.android.com/guide/basics/what-is-android.html|\end{scriptsize}
\bibitem{WikiAnd}
\begin{scriptsize}  \verb|http://en.wikipedia.org/wiki/Android_(operating_system)|\end{scriptsize}
\bibitem{WikiOp}
\begin{scriptsize}  \verb|http://en.wikipedia.org/wiki/Open_Handset_Alliance|\end{scriptsize}
\bibitem{dalvik}
\begin{scriptsize}  \verb|http://en.wikipedia.org/wiki/Dalvik_(software)|\end{scriptsize}
\bibitem{deve}
\begin{scriptsize}  \verb|http://thedevelopersinfo.com/2009/11/17/using-assets-in-android/|\end{scriptsize}
\bibitem{dalvikvm}
\begin{scriptsize}  \verb|http://www.dalvikvm.com/|\end{scriptsize}
\bibitem{soa}
\begin{scriptsize}  \verb|http://developer.android.com/reference/android/util/Log.html| \end{scriptsize}`
\begin{scriptsize}Android in Action, Second Edition, W. Frank Ableson, Robi Sen, Chris King, Manning \end{scriptsize}
\bibitem{beginAnd}
\begin{scriptsize}Beginning Android 2, Mark L. Murphy, Apress \end{scriptsize}
\bibitem{andCook}
\begin{scriptsize}The Android Developer's Cookbook Building Applications with the Android SDK, James Steele, Nelson To, 
Addison-Wesley \end{scriptsize}

\end{thebibliography}

\end{document}
