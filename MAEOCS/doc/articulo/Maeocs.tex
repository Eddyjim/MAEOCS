\documentclass[journal]{IEEEtran}

\usepackage[T1]{fontenc}
\usepackage[utf8]{inputenc}
\usepackage{lmodern}
%\usepackage[spanish,activeacute]{babel}
\usepackage[pdftex]{graphics}   % Para incluir EPS.
%\usepackage{rotating}   % Para rotar tablas.
\usepackage{array}        % Para hacer tablas.
\usepackage{epsfig}
\usepackage{amsmath}      % Debe estar despues de babel !!!
\usepackage{cite}
\usepackage{listings}

\newcommand{\eci}{\textsf{Escuela Colombiana de Ingeniería Julio Garavito}}
\newcommand{\eciu}{\textsf{Escuela Colombiana de Ingeniería}}
\newcommand{\Rlb}{\textsf{Rodrigo López B.}}
\newcommand{\andr}{\texttt{Android~}}
\newcommand{\gp}{\texttt{GPS~}}

\begin{document}

\title{MAEOCS: Mobile Application for Easy Orientation in Confined Spaces}
\date{Octubre 2012}
\author{Carlos~I.~Gaitán,~\IEEEmembership{Estudiante,~\eci,} y Edward~H.~Jiménez,~\IEEEmembership{Estudiante,~\eci.}}

\markboth{MAEOCS, Artículo de Proyecto de Grado, Diciembre 2012}
{Shell \MakeLowercase{\textit{et al.}}: Bare Demo of IEEEtran.cls for Journals}


\maketitle

\begin{abstract}
\end{abstract}

\begin{IEEEkeywords}
Android, Java, GPS.
\end{IEEEkeywords}

\IEEEpeerreviewmaketitle

\section{Introducción}

\IEEEPARstart{H}{}oy en día es fácil evidenciar cómo los planteles públicos y privados van en crecimiento, ampliando sus
zonas en incluso dispersadoras en diferentes puntos de un mismo sector, por ello las personas que son nuevas en estos
ambientes, o que poseen mala memoria, alguna discapacidad cognitiva, o tenga problemas para comunicarse a otros, lo cual les
impide ubicarse de una forma fácil y se ven afectadas cuando pierden el tiempo mientras miran mapas que son poco claros.

Teniendo en cuenta que en esta era tecnológica, los dispositivos móviles cuentan con una capacidad de procesamiento que día
a día aumenta. Las limitaciones que se tenía para resolver problemas de la vida cotidiana a través de un computador se hacen
cada vez menos. El crecimiento en tecnología a impulsado un nuevo modelo de desarrollo de aplicaciones para
dispositivos móviles. Se han desarrollado frameworks y arquitecturas que facilitan el desarrollo e interacción con distintas
tecnologías y estas permiten llevar las ventajas de la computación a la palma de la mano, permitiendo realizar procecimientos y
calculos de casi cualquier complejidad en este tipo de
dispositivos, dando solucíon a problemáticas de muchas índoles.

Como se mencionaba anteriormente, la orientación es un problema que a través de la tecnología movil ya venido siendo tratado.
Actualmente se puede una persona puede saber su posición
geográfica usando un dispositivo que utilize \gp y seleccionando un destino se puede solicitar una ruta trazada que indique como
llegar de un punto a otro. No obstante este tipo de
orientación solo funciona actualmente en espacios abiertos y rutas de transito, dejando de lado lugares que pueden llegar a
desorientar a las personas que transitan en estos sitios.

\section{Planteamiento de la solución}
Para dar solución a esta problemática hay muchas posibles soluciones, hablando de la arquitectura de la solución que se desea usar;
pero el problema de trazar rutas entre dos caminos si se puede llegar a tratar de una manera general para cualquier solución. Partiendo del hecho que calcular una ruta entre todas las conexiones puede representase en modelos abstractos de grafos, que ya han sido parte de investigaciones para solucion de problemas.

La teoría de grafos ha dado como resultado de investigaciones una serie de algoritmos que responden el problema de busqueda de caminos, algoritmos que han sido diseñados para dar solución a distintas estructuras de grafos y a su caracterización. Dentro de los algoritmos se encuentran distintas de estrategias para tratar el problema, lo cual hace a algunos de estos más efectivos en problemas específicos. Tambíen se han diseñado algoritmos que usan patrones expertos para agilizar la toma de desiciones.

Para encontrar la ruta entre dos puntos ubicados en un mapa que podría contener bucles en sus caminos es necesario buscar un algoritmo que pueda avanzar de una forma ágil y que busque reducir la complejidad del calculo, reduciendo tiempo y recursos tecnológicos por la cantidad de datos que se tienen. Teniendo en cuenta estos requerimientos, se encontro un algoritmo que use un patrón experto, una heurística que reduzca la complejidad algoritmica.

\section{Conclusiones}
Ya que \andr es actualmente el sistema operativo para dispositivos moviles mas usado en el mundo, ofrece una ventaja
hablando del numero de usuarios potenciales para una aplicación y el objetivo de la aplicación es ayudar a las personas a
ubicarse en lugares de dificil orientación, hace que el proyecto tenga su foco en la parte social. Esperando producir un
mayor impacto se espera alcanzar la mayor cantidad de personas y por esto que \andr es la plataforma adecuada para hacerlo.
El desarrollo de la aplicación permitío la creación de guias para aprendizaje y desarrollo de aplicaciones en el mismo
sistema, beneficiando no solo a los usuarios de la aplicación sino a su vez a futuros desarrolladores. Durante la
investigación para dar solución del problema de los caminos cortos, se optó por usar un algoritmo de busqueda conocido por
su eficiencia, esto permite que la aplicación se pueda usar en la mayoría de dispositivos, sin necesidad de tener que usar
dispositivos de última tecnologia, y así poder alcanzar mayor cantidad de usuarios.


\renewcommand{\refname}{Referencias}
\begin{thebibliography}{50}
\bibitem{And} Android Developers Official Website.\\
\begin{scriptsize}  \verb|http://developer.android.com/guide/basics/what-is-android.html|\end{scriptsize}
\bibitem{WikiAnd}
\begin{scriptsize}  \verb|http://en.wikipedia.org/wiki/Android_(operating_system)|\end{scriptsize}
\bibitem{WikiOp}
\begin{scriptsize}  \verb|http://en.wikipedia.org/wiki/Open_Handset_Alliance|\end{scriptsize}
\bibitem{dalvik}
\begin{scriptsize}  \verb|http://en.wikipedia.org/wiki/Dalvik_(software)|\end{scriptsize}
\bibitem{deve}
\begin{scriptsize}  \verb|http://thedevelopersinfo.com/2009/11/17/using-assets-in-android/|\end{scriptsize}
\bibitem{dalvikvm}
\begin{scriptsize}  \verb|http://www.dalvikvm.com/|\end{scriptsize}
\bibitem{soa}
\begin{scriptsize}  \verb|http://developer.android.com/reference/android/util/Log.html| \end{scriptsize}`
\begin{scriptsize}Android in Action, Second Edition, W. Frank Ableson, Robi Sen, Chris King, Manning \end{scriptsize}
\bibitem{beginAnd}
\begin{scriptsize}Beginning Android 2, Mark L. Murphy, Apress \end{scriptsize}
\bibitem{andCook}
\begin{scriptsize}The Android Developer's Cookbook Building Applications with the Android SDK, James Steele, Nelson To, 
Addison-Wesley \end{scriptsize}

\end{thebibliography}

\end{document}
