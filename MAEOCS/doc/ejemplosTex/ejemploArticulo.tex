
\documentclass[journal]{IEEEtran}

\usepackage[utf8]{inputenc}
%\usepackage{program}
%\usepackage[spanish,activeacute]{babel}
\usepackage{graphics}   % Para incluir EPS.
%\usepackage{rotating}   % Para rotar tablas.
\usepackage{array}      % Para hacer tablas raras.
\usepackage{epsfig}
%\usepackage{amstex}      % Debe estar despues de babel !!!
\usepackage{amsmath}      % Debe estar despues de babel !!!
\usepackage{cite}
\usepackage{listings}

\newcommand{\eci}{\textsf{Escuela Colombiana de Ingeniería~}}
\newcommand{\eciu}{\textsf{Escuela Colombiana de Ingeniería}}
\newcommand{\Rlb}{\textsf{Rodrigo L'opez B.}}
\newcommand{\pfx}{\textsf{ProofNix~}}
\newcommand{\pfxu}{\textsf{ProofNix}}
\newcommand{\pfs}{\textsf{ProofStar~}}
\newcommand{\pfsu}{\textsf{ProofStar}}
\newcommand{\eclp}{\textsf{Eclipse~}}
\newcommand{\eclpu}{\textsf{Eclipse}}
\newcommand{\mnf}{\texttt{MANIFEST.MF~}}
\newcommand{\mnfu}{\texttt{MANIFEST.MF}}
\newcommand{\plg}{\texttt{plugin.xml~}}
\newcommand{\plgu}{\texttt{plugin.xml}}
\begin{document}
%
% paper title
% can use linebreaks \\ within to get better formatting as desired
\title{Reporte de Investigación sobre Eclipse RCP para el proyecto \pfx RCP (Marzo 2010)}

\author{Oscar~E.~Barrios,~\IEEEmembership{Estudiante,~\eci,}
        Luis~Molina,~\IEEEmembership{Estudiante,~\eci,}}

% The paper headers
\markboth{\pfx~RCP,~Reporte~de~Investigacion, No.~1, Marzo~2010}%
{Shell \MakeLowercase{\textit{et al.}}: Bare Demo of IEEEtran.cls for Journals}

% make the title area
\maketitle


\begin{abstract}
%\boldmath
RCP es una herramienta de implementación de clientes enriquecidos de \eclp basada en el modelo OSGi. Para desarrollar sobre esta arquitectura es necesario llevar  a cabo una enorme 
cantidad de trabajo investigativo y crear ambientes controlados de prueba de tal manera que se inspecciones 
los elementos relevantes para el proyecto. 
\end{abstract}

\begin{IEEEkeywords}
RCP, \eclp, OSGi, prototipos.
\end{IEEEkeywords}

\IEEEpeerreviewmaketitle

\section{Introducción}

\IEEEPARstart{E}{ste} documento es un resumen sobre los avances obtenidos en la investigación sobre 
el ambiente RCP de \eclp. Debido a la dispersa y de alguna manera escasa documentación sobre el tema, el equipo se ha 
dado a la tarea de recolectar, utilizar, decantar y resumir la información adquirida sobre la plataforma. Guiados bajo 
un esquema de creación de prototipos, nos hemos dado a la tarea de construir una aplicación RCP que si bien no le subyace 
ninguna lógica de aplicación, sí resulta útil a la hora de preparar conocimientos y mostrar no sólo las posibilidades de 
la plataforma sino facilitar mediante documentación los usos que se le pueden aplicar enfocados al proyecto \pfx. 

En este primer documento sobre el uso de RCP se explora un producto que se creó con el fin de mostrar de manera concreta 
algunos de los aspectos vitales de la implementación sobre RCP.
El enfoque que le hemos querido dar al trabajo está ligado a las facilidades que entrega \eclp para crear clientes 
RCP, en particular, los \textit{Wizards} que permiten crear, mediante ambientes sencillos y de alguna manera intuitivos, los elementos 
necesarios de la aplicación.
Por esto mismo se proveerán varias imágenes sobre la utilización de los \textit{Wizards}, contrastándolas con la simplicidad que proveen 
frente a la edición sobre los archivos de configuración para los elementos de la interfaz.
Del mismo modo se explicarán de manera general los componentes necesarios para el funcionamiento de la aplicación como lo son 
comandos, vistas, perspectivas y las clases java subyacentes que permiten personalizar la funcionalidad de cada uno de éstos 
elementos, además de las clases directivas Advisors, que tienen un papel fundamental a la hora de ajustar la aplicación según 
se desee.

Se dejarán de lado algunos conceptos que se suponen conocidos por el lector como la arquitectura básica de una aplicación RCP o 
la estructura de plugins para concentrarnos en el contenido mismo de cada uno de los elementos utilizados para darle vida a 
nuestro prototipo.
De la misma manera y en aras de conservar el enfoque que se le quiere dar al desarrollo, se dejarán de lado algunas 
particularidades relacionadas con el código de las clases, centrándonos sólo en los fragmentos modificados o en aquellas partes 
que consideramos pertinentes.


\section{Consideración Sobre el IDE}
Para crear un proyecto de RCP es importante tener en cuenta que el IDE de desarrollo Java no es suficiente; es 
necesario obtener el IDE para desarrollo de \textit{Plug-Ins} o de la misma manera extender el IDE que se tenga. No se explicará acá 
como hacerlo pero se puede obtener el IDE específico para trabajo de \textit{Plug-Ins} en \verb|www.eclipse.org|

\section{El \mnf y \plgu}
Ya que nuestro desarrollo está basado por completo en \textit{Wizards}, hay que entender sobre qué trabajan dichos \textit{Wizards}. Los dos 
componentes esenciales de arquitectura de un Plug-in (la pieza de software mínima instalable y ejecutable de \eclp) son los 
archivos \mnf y \plgu. El primero contiene la descripción de cómo el plugin se describe ante el sistema, el segundo contiene la 
descripción de las extensiones (componentes que se agregan al \textit{Plug-In}) y los puntos de extensión (lugares donde el desarrollador 
permite conectar otras extensiones). 

Los \textit{Wizards} de \eclp simplifican la manipulación de dichos archivos y contribuyen 
automatizando algunas tareas dispendiosas que alejan al desarrollador de la visión general de lo que busca construir sin decantar 
ni sobresimplificar.
Para acceder a dichos \textit{Wizards}, basta con hacer doble click en \mnf en la vista de exploración del proyecto, ubicado en el 
directorio \texttt{META-INF} del proyecto.

\section{Perspectivas Y Vistas}
Una \textbf{Perspectiva} \cite{SiVl} \textit{(Perspective)} es definida como el conjunto inicial de herramientas y vistas en el \textbf{Workbench}, son las 
perspectivas las que proveen la estructura general de una tarea de desarrollo.

Una \textbf{Vista}\cite{SiVl} \textit{(View)} es la estructura de ventana o de barra lateral utilizada para la mayoría de tareas del Workbench 
y cuya función principal es presentar información de alguna manera al usuario. A ellas se pueden enlazar varios tipos de 
elementos como exploradores, editores, tableros de comando y demás, haciéndolas un componente muy flexible y versátil.

Vamos a mostrar el procedimiento de creación de una vista:
\begin{enumerate}
\item Acceder a \mnf.
\item En la pestaña de \textit{Extensiones}, seleccionar \textit{Add}...
\item En el cuadro de diálogo escribir \textit{View}
\item De las alternativas mostradas seleccionar \texttt{org.eclipse.ui.views}
\item Click en \textit{Finish}.
\end{enumerate}

Desde ésta extensión podemos agregar las vistas que necesitemos. Una vista es única para todo el producto, no existen varias "instancias" 
de la misma vista, esto significa que si una vista está contenida en varias perspectivas, será la misma vista, mostrando los mismos elementos siempre.
Para agregar la vista nos ubicamos en la pestaña \textit{Extensions} y:
\begin{enumerate}
 \item Click derecho en \texttt{org.eclipse.ui.views} 
 \item \textit{New} $>$ \textit{View}
\end{enumerate}
Los atributos de la vista que utilizaremos son:
\begin{itemize}
 \item id: es indispensable seleccionar un identificador único para cada elemento del \textit{Plug-In}, por buena práctica haremos que coincida 
con el nombre del componente siempre, de este modo, si el campo name es MainView, y nuestro proyecto se llama eci.edu.pgr1.sandboxrcp, 
entonces el id de la vista debe ser \texttt{eci.edu.pgr1.sandboxrcp.views.mainView}.
\item name: El nombre que le daremos a la vista, para identificarla como usuario, siguiendo el ejemplo: MainView
\item class: es la clase que maneja la lógica de la vista, el nombre es autogenerado, y debemos es crear la clase, para ello hacemos 
click en el hipervínculo class, que como resultado nos mostrará la ventana de creación de la clase correspondiente clase Java.
\end{itemize}
Los demás atributos no se tocarán en este documento.
\\
Ahora creemos alguna perspectiva para añadir las vistas recién creadas. El prototipo que hemos creado cuenta con dos perspectivas, una 
principal, construida por defecto usualmente, y una auxiliar, con el fin demostrar algunos cambios a la hora de establecer la distribución 
de las vistas que contiene.
La perspectiva se crea accediendo a \mnf, y, sobre la pestaña de Extensions, hacer click en Add, escribir perspective en el campo de búsqueda 
y seleccionar \texttt{org.eclipse.ui.perspectives}. Si ya existe la extensión, podemos continuar desde el siguiente párrafo. Esto creará una nueva extensión 
de perspectivas bajo la cual subordinaremos las perspectivas que incluyamos de ahora en adelante.
Una vez hecho esto, hacemos click derecho sobre la extensión recién creada, navegamos a New y seleccionamos \textit{Perspective}. Debemos proporcionar un \textit{id}, 
\textit{name} y \textit{class} de la misma manera como se hizo con el View.

\section{Comandos}
La interacción usual de cualquier interfaz suele ser basada en botones. RCP permite desplegar una serie de comandos que dependiendo de su contexto 
se comportan de maneras distintas, haciéndolos versátiles y sencillos de implementar.
Lars Vogela \cite{VoLa} describe los comandos como una descripción declarativa de un componente que es independiente de los detalles de implementación. 
Un \textit{command} puede ser categorizado y se le puede asignar una tecla al \textit{command}. Pueden ser además utilizados en \textit{menus}, \textit{toolbars} y/o menús contextuales.
El procedimiento de creación de comandos es muy similar a los ya creados, agregando algunos pasos adicionales.
Para crear un \textit{command} accedemos a la pestaña \textit{Extensions} de \mnf, agregamos una extensión, buscamos \textit{command} y agregamos \texttt{org.eclipse.ui.commands}. 
Hacemos click derecho sobre la extensión y agregamos command. Los comandos necesitan un id, un nombre y en particular un \textit{Handler}, que será la clase 
encargada de manejar la lógica del \textit{command}.
\begin{center}
 \begin{figure}
 \centering
 \includegraphics[bb=0 0 255 179]{./imagenes/figura1.jpg}
 % figura1.jpg: 340x238 pixel, 96dpi, 9.00x6.30 cm, bb=0 0 255 179
 \caption{Panel \textit{Extensions} con los distintos comandos utilizados en el prototipo, todo \textit{command} debe especificar un id, un nombre y un handler que 
	  describe su lógica interna.}
\end{figure}
\end{center}

\begin{center}
\begin{tabular}{|l|c|l|}\hline
bb & gg & hh\\
nn & mm & ñ\\\hline
\end{tabular}
\end{center}


Creemos el \textit{command} Exit entonces, encargado de salir de la aplicación. Para ello, una vez creado el \textit{command}:
\begin{enumerate}
 \item Escribimos en su id: \texttt{eci.edu.sandboxrcp.commands.Exit}. 
 \item En el campo name ingresamos \textit{Exit}
 \item En el campo defaultHandler  hacemos click en el hipervínculo y creamos la clase \texttt{CommandExit}.
 \item Asegurémonos que dicha clase implemente \texttt{IHandler} y extienda a clase \texttt{AbstractHandler}.
 \item El código de la clase \texttt{CommandExit} debe ser:
\end{enumerate}

\lstset{language=Java, tabsize=4}
\begin{scriptsize}\ttfamily\begin{lstlisting}
package eci.edu.pgr1.sandboxrcp.commands;

import org.eclipse.core.commands.AbstractHandler;
import org.eclipse.core.commands.ExecutionEvent;
import org.eclipse.core.commands.ExecutionException;
import org.eclipse.core.commands.IHandler;
import org.eclipse.ui.handlers.HandlerUtil;

public class CommandExit extends AbstractHandler 
  implements IHandler {

    @Override
    public Object execute(ExecutionEvent event) 
    throws ExecutionException {
	  HandlerUtil.getActiveWorkbenchWindow(event).
	    close();
	  return null;
  }
}
\end{lstlisting}\end{scriptsize}

\section{Advisors}
Los \textit{Advisors} son parte esencial de RCP, pues son los encargados de la configuración general del \textit{Workbench}, la ventana del \textit{Workbench} y la barra de acciones. 
El proceso es iniciado dentro del programa principal con una llamada a \texttt{PlatformUI.createAndRunWorkbench (Display, ApplicationWorkbenchAdvisor)}. Son tres clases: 
\texttt{WorkbenchAdvisor}, \texttt{WorkbenchWindowAdvisor}, y \texttt{ActionBarAdvisor}.\cite{SiVl}
En el prototipo, personalizamos el ApplicationWorkbenchWindowAdvisor, este elemento se crea una vez por cada ventana de \textit{Workbench} y se utiliza para configurar 
la ventana. Una aplicación puede utilizar algunos métodos para cambar la configuración de la ventana. Nuestra configuración es la siguiente (los comentarios describen 
la funcionalidad de cada línea:  

\lstset{language=Java, tabsize=4}
\begin{scriptsize}\ttfamily\begin{lstlisting}
public void preWindowOpen() {
    IWorkbenchWindowConfigurer configurer =
      getWindowConfigurer();
    //Tamano inicial de la ventana
    configurer.setInitialSize(new Point(400, 300));
    //Mostrar la barra de herramientas, llamada Coolbar
    configurer.setShowCoolBar(true);
    //Mostrar la barra de cambio de perspectivas
    configurer.setShowPerspectiveBar(true);
    //Mostrar la barra de menu
    configurer.setShowMenuBar(true);
    //Mostrar la barra de status
    configurer.setShowStatusLine(false);
    
    IPreferenceStore apiStore = PlatformUI.
      getPreferenceStore();
    apiStore.setValue(
      IWorkbenchPreferenceConstants.DOCK_PERSPECTIVE_BAR,
      "TOP_LEFT");
}
\end{lstlisting}\end{scriptsize}

\section{Integrando los componentes}
Ahora integremos nuestros componentes. El prototipo que hemos creado cuenta con dos perspectivas, una principal, construida por defecto usualmente, y una auxiliar, 
con el fin demostrar algunos cambios a la hora de establecer la distribución de las vistas que contiene.
Para ello añadimos en primera instancia las vistas que queramos a la perspectiva correspondiente. Por ejemplo, agregar el \texttt{MainView} creado anteriormente es muy sencillo, 
sólo falta añadirlo mediante código a la perspectiva de la siguiente manera (adicionalmente agregamos una consola, para contrastar con la vista, nótese los atributos 
de posición relativa: 

\lstset{language=Java, tabsize=4}
\begin{scriptsize}\ttfamily\begin{lstlisting}
public void createInitialLayout(IPageLayout layout) {
    layout.setEditorAreaVisible(false);
    String editorArea = layout.getEditorArea();  
    // Consola
    layout.addStandaloneView(
      IConsoleConstants.ID_CONSOLE_VIEW, true, 
      IPageLayout.BOTTOM, 0.95f, editorArea);
    //Agregar el MAin VIew
    layout.
      addStandaloneView(
	"eci.edu.pgr1.sandboxrcp.views.mainView", 
      true, IPageLayout.LEFT,  0.75f, editorArea);
}

\end{lstlisting}\end{scriptsize}
Ahora agreguemos el botón a un \textit{menu contributor}, que no es más que un contenedor dentro del cual pueden albergarse otros menús o botones. Dicho \textit{menu contributor} será 
agregado a nuestro \textit{MainView} que ya pertenece a la perspectiva predeterminada.
El \textit{menu contributor} se agrega de la siguiente manera:

\begin{enumerate}
 \item En \mnf accedemos a \textit{Extensions}.
 \item En Extensions damos click en \textit{Add}...
 \item Escribimos \textit{menu} y seleccionamos \texttt{eclipse.ui.menu}.
 \item En \textit{locationURI} ponemos \texttt{menu:org.eclipse.ui.main.menu}.
 \item Damos click derecho sobre nuestro \textit{contributor} y damos click en \textit{New$>$menu}.
 \item Ponemos en el campo \textit{label}: \textit{File}.
 \item Hacemos click derecho en nuestro \textit{menu File}, \textit{New $>$ command}.
 \item En éste \textit{Wizard} vamos a enlazar el \textit{command Exit} que previamente definimos a un botón del \textit{menu File}. Frente \textit{commandId} seleccionamos \textit{Browse}, 
 escribimos \texttt{eci.edu.pgr1.sandboxrcp.commands.Exit} y seleccionamos la entrada.
 \item En label ingresamos \textit{Exit}.
\end{enumerate}
Éste procedimiento agrega un menú en la barra principal de nuestro \textit{Workbench}, con un submenú llamado \textit{File} que tiene una opción \textit{Exit}. Dicho botón sale de la aplicación.
El \textit{URI} que ingresamos en el menu contributor es una especificación que define el punto de inserción en el cual se añaden sus elementos, de tal manera que se tenga una 
"ruta" del contenedor dentro del cual se van a añadir componentes.
Normalmente el \textit{URI} consta de tres partes, la primera es el esquema, que tipo de componente administra la contribución. El \textit{ID} es típicamente el tipo de contribución y 
finalmente se puede realizar una consulta sobre la posición del componente.

\renewcommand{\refname}{Bibliografía}
\begin{thebibliography}{50}
\bibitem{SiVl} Vladimir Silva \textit{Eclipse Rich Client Platform Projects}. APress 2009.
\bibitem{VoLa} Lars Vogel. \textit{Eclipse RCP - Tutorial (Eclipse 3.5)}\\
\begin{scriptsize}  \verb|http://www.vogella.de/articles/RichClientPlatform/article.html|\end{scriptsize}
\end{thebibliography}

\end{document}


